%%========================================================================
%% LaTeX scriptiesjabloon
%%========================================================================


%%========================================================================
%% Preamble
%%========================================================================

\documentclass[pdftex,a4paper,12pt,twoside]{report}

%%---------- Extra functionaliteit ---------------------------------------

\usepackage[utf8]{inputenc}  % Accenten gebruiken in tekst (vb. é ipv \'e)
\usepackage{amsfonts}        % AMS math packages: extra wiskundige
\usepackage{amsmath}         %   symbolen (o.a. getallen-
\usepackage{amssymb}         %   verzamelingen N, R, Z, Q, etc.)
\usepackage[UKenglish]{babel}    % Taalinstellingen: woordsplitsingen,
                             %  commando's voor speciale karakters
                             %  ("dutch" voor NL)
							 %  ("UKenglish" voor brits engels)
\usepackage{eurosym}         % Euro-symbool €
\usepackage{graphicx}        % Invoegen van tekeningen
\usepackage[pdftex,bookmarks=true]{hyperref}
                             % PDF krijgt klikbare links & verwijzingen,
                             %  inhoudstafel
\usepackage{listings}        % Broncode mooi opmaken
\usepackage{multirow}        % Tekst over verschillende cellen in tabellen
\usepackage{rotating}        % Tabellen en figuren roteren
\usepackage{natbib}          % Betere bibliografiestijlen
\usepackage{fancyhdr}        % Pagina-opmaak met hoofd- en voettekst

%%---------- Layout ------------------------------------------------------

% hoofdingen, enz.
\pagestyle{fancy}

% lijn, wordt gebruikt in titelpagina
\newcommand{\HRule}{\rule{\linewidth}{0.5mm}}

% Leeg blad
\newcommand{\emptypage}{
\newpage
\thispagestyle{empty}
\mbox{}
\newpage
}
 
% Gebruik een schreefloos lettertype ipv het "oubollig" uitziende
% Computer Modern
\renewcommand{\familydefault}{\sfdefault}     

% Commando voor invoegen Java-broncodebestanden (dank aan Niels Corneille)
% Gebruik: \codefragment{source/MijnKlasse.java}{Uitleg bij de code}
\newcommand{\codefragment}[2]{ \lstset{%
  language=java,
  breaklines=true,
  float=th,
  caption={#2},
  basicstyle=\scriptsize,
  frame=single
}
\lstinputlisting{#1}}

%%---------- Documenteigenschappen ---------------------------------------
%% Vul dit aan met je eigen info:

% Je eigen naam
\newcommand{\studenta}{Pieter {Van Eeckhout}}

% Eventueel naam van een medestudent.
% Laat in commentaar indien niet van toepassing
%\newcommand{\studentb}{Steven Stevens}

% De naam van je stage-/bachelorproefbegeleider
\newcommand{\begeleider}{Johan {Van Schoor}} 

% De naam (én firma/organisatie) van je mentor/promotor
% Laat in commentaar indien niet van toepassing
%\newcommand{\mentor}{Jan Janssen, ACME Inc.}

% De titel van je scriptie/stageverslag
\newcommand{\titel}{Solving CAPTCHA using neural networks}

% Ondertitel
\newcommand{\ondertitel}{}

% Datum van indienen
\newcommand{\datum}{31 mei 2013}

% Academiejaar
\newcommand{\academiejaar}{2012-2013}

%%========================================================================
%% Inhoud document
%%========================================================================

\begin{document}

%%---------- Front matter ------------------------------------------------
%% Het voorblad - Hier moet je in principe niets wijzigen.

\begin{titlepage}
\begin{center}
\includegraphics[width=4cm]{../FBO-NL}\\[.5cm]

Professionele Bachelor toegepaste informatica\\
Academiejaar \academiejaar

\vfill

\HRule \\[0.4cm]
{ \huge \bfseries \titel}\\[0.4cm]
\HRule \\[0.4cm]

{\Large \ondertitel}\\[0.4cm]

Ingediend op \datum

\vfill

% Studenten en begeleiders
\begin{minipage}{0.49\textwidth}
\begin{flushleft}
\emph{Student\ifdefined\studentb en\fi :}\\
\studenta \\
\ifdefined\studentb \studentb \fi\par
\end{flushleft}
\end{minipage}
\begin{minipage}{0.49\textwidth}
\begin{flushright}
\emph{Begeleider:}\\ \begeleider\\
\ifdefined\mentor \emph{Mentor:}\\ \mentor \fi
\end{flushright}
\end{minipage}

\end{center}

\end{titlepage}

% Schutblad

\emptypage

% Herhaling titelblad

\begin{titlepage}
\begin{center}
HoGent Bedrijf en Organisatie\\
Professionele Bachelor toegepaste informatica\\
Academiejaar \academiejaar

\vfill

\HRule \\[0.4cm]
{ \huge \bfseries \titel}\\[0.4cm]
\HRule \\[0.4cm]

{\Large \ondertitel}\\[0.4cm]

Ingediend op \datum

\vfill

% Studenten en begeleiders
\begin{minipage}{0.49\textwidth}
\begin{flushleft}
\emph{Student\ifdefined\studentb en\fi :}\\
\studenta \\
\ifdefined\studentb \studentb \fi\par
\end{flushleft}
\end{minipage}
\begin{minipage}{0.49\textwidth}
\begin{flushright}
\emph{Begeleider:}\\ \begeleider\\
\ifdefined\mentor \emph{Mentor:}\\ \mentor \fi
\end{flushright}
\end{minipage}

\end{center}

\end{titlepage}

%% Inhoudstafel

\tableofcontents

%%---------- Kern --------------------------------------------------------

\begin{abstract}

%% De "abstract" of samenvatting is een kernachtige (max 1 blz. voor een
% thesis) synthese van het document. In ons geval beschrijf je kort de
% probleemstelling en de context, de onderzoeksvragen, de aanpak en de
% resultaten. 

\end{abstract}

\chapter*{Voorwoord}
\label{ch:voorwoord}

%% Vergeet ook niet te bedankten wie je geholpen/gesteund/... heeft


\chapter{Inleiding}
\label{ch:inleiding}

%% Je kan de titel best vervangen door een concrete verwijzing naar
% het onderzoeksdomein, bv. "State of the art in productie-scheduling"

% Gebruik verwijzingen naar de bibliografie (\citep{} of \cite{})

Voorbeeld van verwijzen naar literatuur. Als je expliciet naar de auteur verwijst in de zin, gebruik je \texttt{$\backslash${}cite\{\}}.
Soms wil je de auteur niet expliciet vernoemen, dan gebruik je \texttt{$\backslash${}citep\{\}}.

\cite{Knuth1998} schreef een van de standaardwerken over sorteer- en zoekalgoritmen. \emph{Unified Process} is een populair raamwerk voor iteratieve en incrementele software-ontwikkelingsprocessen~\citep{Larman2004}.

\cite{Bartkowiak2004} test citation

\chapter{Probleemstelling en onderzoeksvragen}
\label{ch:probleemstelling}

\section{Probleemstelling}
\label{sec:probleemstelling}

\section{Onderzoeksvragen}
\label{sec:onderzoeksvragen}

\chapter{Methodologie}
\label{ch:methodologie}

\chapter{Corpus}
\label{ch:corpus}

%% TODO: de structuur en titel van deze hoofdstukken hangen af van je
% eigen onderzoek. Kies gepaste titel.

%% 
%% Conclusie
%% 

\chapter{Conclusie}
\label{ch:conclusie}

%%---------- Bijlagen ----------------------------------------------------

\appendix

\chapter{Broncode}
\label{ch:broncode}

% Automatisch invoegen van al je Java broncode:
% 1/ maak een link naar je broncodedirectory naar subdirectory source
%      ln -s /path/to/java/src/ ./source
%    Of kopieer desnoods al je broncodebestanden. Zorg dat je
%    versiebeheersysteem deze directory negeert!
% 2/ Genereer source.tex met het script source.sh
%      ./source.sh
% 3/ Haal volgende regel uit commentaar
%
\section{Package be}

\section{Package be.hogent}

\section{Package be.hogent.bulksolvingstatistics}

\section{Package be.hogent.captchabuilder}

\section{Package be.hogent.captchacleanup}

\section{Package be.hogent.captchasolvingnetwork}
\codefragment{source/be/hogent/bulksolvingstatistics/BulkSolvingStatistics.java}{be.hogent.bulksolvingstatistics.BulkSolvingStatistics}

\section{Package be.hogent.bulksolvingstatistics.domain}

\section{Package be.hogent.bulksolvingstatistics.persistance}

\section{Package be.hogent.bulksolvingstatistics.ui}

\section{Package be.hogent.captchabuilder.builder}

\section{Package be.hogent.captchabuilder.elementcreator}

\section{Package be.hogent.captchabuilder.util}
\codefragment{source/be/hogent/captchacleanup/CaptchaCleanup.java}{be.hogent.captchacleanup.CaptchaCleanup}

\section{Package be.hogent.captchacleanup.utils}
\codefragment{source/be/hogent/captchasolvingnetwork/CaptchaSolvingNetwork.java}{be.hogent.captchasolvingnetwork.CaptchaSolvingNetwork}

\section{Package be.hogent.captchasolvingnetwork.encog_2}

\section{Package be.hogent.captchasolvingnetwork.network}

\section{Package be.hogent.captchasolvingnetwork.util}
\codefragment{source/be/hogent/bulksolvingstatistics/domain/DomainFacade.java}{be.hogent.bulksolvingstatistics.domain.DomainFacade}

\section{Package be.hogent.bulksolvingstatistics.domain.neuralnetwork}
\codefragment{source/be/hogent/bulksolvingstatistics/persistance/DatabaseConnection.java}{be.hogent.bulksolvingstatistics.persistance.DatabaseConnection}

\section{Package be.hogent.bulksolvingstatistics.persistance.mappers}
\codefragment{source/be/hogent/bulksolvingstatistics/persistance/PersistanceController.java}{be.hogent.bulksolvingstatistics.persistance.PersistanceController}
\codefragment{source/be/hogent/bulksolvingstatistics/ui/BulkSolvingStatisticsGui.java}{be.hogent.bulksolvingstatistics.ui.BulkSolvingStatisticsGui}
\codefragment{source/be/hogent/captchabuilder/builder/BackgroundParser.java}{be.hogent.captchabuilder.builder.BackgroundParser}
\codefragment{source/be/hogent/captchabuilder/builder/BorderParser.java}{be.hogent.captchabuilder.builder.BorderParser}
\codefragment{source/be/hogent/captchabuilder/builder/Captcha.java}{be.hogent.captchabuilder.builder.Captcha}
\codefragment{source/be/hogent/captchabuilder/builder/CaptchaBuilder.java}{be.hogent.captchabuilder.builder.CaptchaBuilder}
\codefragment{source/be/hogent/captchabuilder/builder/CaptchaBuildSequenceParser.java}{be.hogent.captchabuilder.builder.CaptchaBuildSequenceParser}
\codefragment{source/be/hogent/captchabuilder/builder/ColorsParser.java}{be.hogent.captchabuilder.builder.ColorsParser}
\codefragment{source/be/hogent/captchabuilder/builder/GimpyParser.java}{be.hogent.captchabuilder.builder.GimpyParser}
\codefragment{source/be/hogent/captchabuilder/builder/NoiseParser.java}{be.hogent.captchabuilder.builder.NoiseParser}
\codefragment{source/be/hogent/captchabuilder/builder/TextParser.java}{be.hogent.captchabuilder.builder.TextParser}
\codefragment{source/be/hogent/captchabuilder/elementcreator/CaptchaElementCreatorBuilder.java}{be.hogent.captchabuilder.elementcreator.CaptchaElementCreatorBuilder}

\section{Package be.hogent.captchabuilder.elementcreator.producer}

\section{Package be.hogent.captchabuilder.elementcreator.renderer}
\codefragment{source/be/hogent/captchabuilder/util/ArrayUtil.java}{be.hogent.captchabuilder.util.ArrayUtil}
\codefragment{source/be/hogent/captchabuilder/util/CaptchaDAO.java}{be.hogent.captchabuilder.util.CaptchaDAO}
\codefragment{source/be/hogent/captchabuilder/util/ColorRangeRGBA.java}{be.hogent.captchabuilder.util.ColorRangeRGBA}

\section{Package be.hogent.captchabuilder.util.enums}
\codefragment{source/be/hogent/captchabuilder/util/ImageUtil.java}{be.hogent.captchabuilder.util.ImageUtil}
\codefragment{source/be/hogent/captchacleanup/utils/ImageToArray.java}{be.hogent.captchacleanup.utils.ImageToArray}
\codefragment{source/be/hogent/captchacleanup/utils/ImageUtils.java}{be.hogent.captchacleanup.utils.ImageUtils}

\section{Package be.hogent.captchacleanup.utils.textfromimage}
\codefragment{source/be/hogent/captchasolvingnetwork/encog_2/EncogHopfieldNetworkExample.java}{be.hogent.captchasolvingnetwork.encog_2.EncogHopfieldNetworkExample}

\section{Package be.hogent.captchasolvingnetwork.network.encog}
\codefragment{source/be/hogent/captchasolvingnetwork/network/NeuralNetwork.java}{be.hogent.captchasolvingnetwork.network.NeuralNetwork}
\codefragment{source/be/hogent/captchasolvingnetwork/network/NeuralNetworkActions.java}{be.hogent.captchasolvingnetwork.network.NeuralNetworkActions}

\section{Package be.hogent.captchasolvingnetwork.network.neuroph}
\codefragment{source/be/hogent/captchasolvingnetwork/util/CharacterPatternUtils.java}{be.hogent.captchasolvingnetwork.util.CharacterPatternUtils}
\codefragment{source/be/hogent/captchasolvingnetwork/util/EncogTrainingSet.java}{be.hogent.captchasolvingnetwork.util.EncogTrainingSet}
\codefragment{source/be/hogent/captchasolvingnetwork/util/ImageToInputPattern.java}{be.hogent.captchasolvingnetwork.util.ImageToInputPattern}

\section{Package be.hogent.bulksolvingstatistics.domain.neuralnetwork.dataobjects}
\codefragment{source/be/hogent/bulksolvingstatistics/domain/neuralnetwork/DefaultNeuralNetworkController.java}{be.hogent.bulksolvingstatistics.domain.neuralnetwork.DefaultNeuralNetworkController}
\codefragment{source/be/hogent/bulksolvingstatistics/domain/neuralnetwork/DefaultNeuralNetworkRepository.java}{be.hogent.bulksolvingstatistics.domain.neuralnetwork.DefaultNeuralNetworkRepository}

\section{Package be.hogent.bulksolvingstatistics.domain.neuralnetwork.encogutils}
\codefragment{source/be/hogent/bulksolvingstatistics/domain/neuralnetwork/NeuralNetworkController.java}{be.hogent.bulksolvingstatistics.domain.neuralnetwork.NeuralNetworkController}
\codefragment{source/be/hogent/bulksolvingstatistics/domain/neuralnetwork/NeuralNetworkRepository.java}{be.hogent.bulksolvingstatistics.domain.neuralnetwork.NeuralNetworkRepository}
\codefragment{source/be/hogent/bulksolvingstatistics/persistance/mappers/Mapper.java}{be.hogent.bulksolvingstatistics.persistance.mappers.Mapper}
\codefragment{source/be/hogent/bulksolvingstatistics/persistance/mappers/NeuralNetworkMapper.java}{be.hogent.bulksolvingstatistics.persistance.mappers.NeuralNetworkMapper}
\codefragment{source/be/hogent/bulksolvingstatistics/persistance/mappers/TestResultMapper.java}{be.hogent.bulksolvingstatistics.persistance.mappers.TestResultMapper}

\section{Package be.hogent.captchabuilder.elementcreator.producer.background}

\section{Package be.hogent.captchabuilder.elementcreator.producer.border}

\section{Package be.hogent.captchabuilder.elementcreator.producer.noise}

\section{Package be.hogent.captchabuilder.elementcreator.producer.text}

\section{Package be.hogent.captchabuilder.elementcreator.renderer.gimpy}

\section{Package be.hogent.captchabuilder.elementcreator.renderer.text}
\codefragment{source/be/hogent/captchabuilder/util/enums/CaptchaConstants.java}{be.hogent.captchabuilder.util.enums.CaptchaConstants}

\section{Package be.hogent.captchabuilder.util.enums.producer}

\section{Package be.hogent.captchabuilder.util.enums.renderer}
\codefragment{source/be/hogent/captchacleanup/utils/textfromimage/GetImageText.java}{be.hogent.captchacleanup.utils.textfromimage.GetImageText}
\codefragment{source/be/hogent/captchacleanup/utils/textfromimage/TextRegion.java}{be.hogent.captchacleanup.utils.textfromimage.TextRegion}
\codefragment{source/be/hogent/captchasolvingnetwork/network/encog/EncogBasicNetwork.java}{be.hogent.captchasolvingnetwork.network.encog.EncogBasicNetwork}
\codefragment{source/be/hogent/captchasolvingnetwork/network/encog/EncogBasicNetworkBuilder.java}{be.hogent.captchasolvingnetwork.network.encog.EncogBasicNetworkBuilder}
\codefragment{source/be/hogent/captchasolvingnetwork/network/encog/EncogHopfieldNetwork.java}{be.hogent.captchasolvingnetwork.network.encog.EncogHopfieldNetwork}
\codefragment{source/be/hogent/captchasolvingnetwork/network/encog/EncogHopfieldNetworkBuilder.java}{be.hogent.captchasolvingnetwork.network.encog.EncogHopfieldNetworkBuilder}

\section{Package be.hogent.captchasolvingnetwork.network.encog.util}
\codefragment{source/be/hogent/bulksolvingstatistics/domain/neuralnetwork/dataobjects/NeuralNetworkDataObject.java}{be.hogent.bulksolvingstatistics.domain.neuralnetwork.dataobjects.NeuralNetworkDataObject}
\codefragment{source/be/hogent/bulksolvingstatistics/domain/neuralnetwork/dataobjects/NeuralNetworkDataObjectBuilder.java}{be.hogent.bulksolvingstatistics.domain.neuralnetwork.dataobjects.NeuralNetworkDataObjectBuilder}
\codefragment{source/be/hogent/bulksolvingstatistics/domain/neuralnetwork/dataobjects/TestResultDataObject.java}{be.hogent.bulksolvingstatistics.domain.neuralnetwork.dataobjects.TestResultDataObject}
\codefragment{source/be/hogent/bulksolvingstatistics/domain/neuralnetwork/dataobjects/TestResultDataObjectBuilder.java}{be.hogent.bulksolvingstatistics.domain.neuralnetwork.dataobjects.TestResultDataObjectBuilder}
\codefragment{source/be/hogent/bulksolvingstatistics/domain/neuralnetwork/encogutils/EncogTrainingSet.java}{be.hogent.bulksolvingstatistics.domain.neuralnetwork.encogutils.EncogTrainingSet}
\codefragment{source/be/hogent/captchabuilder/elementcreator/producer/background/AbstractBackgroundProducer.java}{be.hogent.captchabuilder.elementcreator.producer.background.AbstractBackgroundProducer}
\codefragment{source/be/hogent/captchabuilder/elementcreator/producer/background/BackgroundProducer.java}{be.hogent.captchabuilder.elementcreator.producer.background.BackgroundProducer}
\codefragment{source/be/hogent/captchabuilder/elementcreator/producer/background/BackgroundProducerBuilder.java}{be.hogent.captchabuilder.elementcreator.producer.background.BackgroundProducerBuilder}
\codefragment{source/be/hogent/captchabuilder/elementcreator/producer/background/FlatColorBackgroundProducer.java}{be.hogent.captchabuilder.elementcreator.producer.background.FlatColorBackgroundProducer}
\codefragment{source/be/hogent/captchabuilder/elementcreator/producer/background/SquigglesBackgroundProducer.java}{be.hogent.captchabuilder.elementcreator.producer.background.SquigglesBackgroundProducer}
\codefragment{source/be/hogent/captchabuilder/elementcreator/producer/background/TransparentBackgroundProducer.java}{be.hogent.captchabuilder.elementcreator.producer.background.TransparentBackgroundProducer}
\codefragment{source/be/hogent/captchabuilder/elementcreator/producer/background/TwoColorGradientBackgroundProducer.java}{be.hogent.captchabuilder.elementcreator.producer.background.TwoColorGradientBackgroundProducer}
\codefragment{source/be/hogent/captchabuilder/elementcreator/producer/border/AbstractBorderProducer.java}{be.hogent.captchabuilder.elementcreator.producer.border.AbstractBorderProducer}
\codefragment{source/be/hogent/captchabuilder/elementcreator/producer/border/BorderProducer.java}{be.hogent.captchabuilder.elementcreator.producer.border.BorderProducer}
\codefragment{source/be/hogent/captchabuilder/elementcreator/producer/border/BorderProducerBuilder.java}{be.hogent.captchabuilder.elementcreator.producer.border.BorderProducerBuilder}
\codefragment{source/be/hogent/captchabuilder/elementcreator/producer/border/SolidBorderProducer.java}{be.hogent.captchabuilder.elementcreator.producer.border.SolidBorderProducer}
\codefragment{source/be/hogent/captchabuilder/elementcreator/producer/noise/AbstractNoiseProducer.java}{be.hogent.captchabuilder.elementcreator.producer.noise.AbstractNoiseProducer}
\codefragment{source/be/hogent/captchabuilder/elementcreator/producer/noise/CurvedLineNoiseProducer.java}{be.hogent.captchabuilder.elementcreator.producer.noise.CurvedLineNoiseProducer}
\codefragment{source/be/hogent/captchabuilder/elementcreator/producer/noise/NoiseProducer.java}{be.hogent.captchabuilder.elementcreator.producer.noise.NoiseProducer}
\codefragment{source/be/hogent/captchabuilder/elementcreator/producer/noise/NoiseProducerBuilder.java}{be.hogent.captchabuilder.elementcreator.producer.noise.NoiseProducerBuilder}
\codefragment{source/be/hogent/captchabuilder/elementcreator/producer/noise/StraightLineNoiseProducer.java}{be.hogent.captchabuilder.elementcreator.producer.noise.StraightLineNoiseProducer}
\codefragment{source/be/hogent/captchabuilder/elementcreator/producer/text/AbstractTextProducer.java}{be.hogent.captchabuilder.elementcreator.producer.text.AbstractTextProducer}
\codefragment{source/be/hogent/captchabuilder/elementcreator/producer/text/AlphanumericTextProducer.java}{be.hogent.captchabuilder.elementcreator.producer.text.AlphanumericTextProducer}
\codefragment{source/be/hogent/captchabuilder/elementcreator/producer/text/ArabicTextProducer.java}{be.hogent.captchabuilder.elementcreator.producer.text.ArabicTextProducer}
\codefragment{source/be/hogent/captchabuilder/elementcreator/producer/text/ChineseTextProducer.java}{be.hogent.captchabuilder.elementcreator.producer.text.ChineseTextProducer}
\codefragment{source/be/hogent/captchabuilder/elementcreator/producer/text/LetterTextProducer.java}{be.hogent.captchabuilder.elementcreator.producer.text.LetterTextProducer}
\codefragment{source/be/hogent/captchabuilder/elementcreator/producer/text/NumbersProducer.java}{be.hogent.captchabuilder.elementcreator.producer.text.NumbersProducer}
\codefragment{source/be/hogent/captchabuilder/elementcreator/producer/text/ReducedAlphanumericTextProducer.java}{be.hogent.captchabuilder.elementcreator.producer.text.ReducedAlphanumericTextProducer}
\codefragment{source/be/hogent/captchabuilder/elementcreator/producer/text/SpecialAlphanumericTextProducer.java}{be.hogent.captchabuilder.elementcreator.producer.text.SpecialAlphanumericTextProducer}
\codefragment{source/be/hogent/captchabuilder/elementcreator/producer/text/SpecialLetterTextProducer.java}{be.hogent.captchabuilder.elementcreator.producer.text.SpecialLetterTextProducer}
\codefragment{source/be/hogent/captchabuilder/elementcreator/producer/text/SpecialNumbersProducer.java}{be.hogent.captchabuilder.elementcreator.producer.text.SpecialNumbersProducer}
\codefragment{source/be/hogent/captchabuilder/elementcreator/producer/text/TextProducer.java}{be.hogent.captchabuilder.elementcreator.producer.text.TextProducer}
\codefragment{source/be/hogent/captchabuilder/elementcreator/producer/text/TextProducerBuilder.java}{be.hogent.captchabuilder.elementcreator.producer.text.TextProducerBuilder}
\codefragment{source/be/hogent/captchabuilder/elementcreator/renderer/gimpy/AbstractGimpyRenderer.java}{be.hogent.captchabuilder.elementcreator.renderer.gimpy.AbstractGimpyRenderer}
\codefragment{source/be/hogent/captchabuilder/elementcreator/renderer/gimpy/BlockGimpyRenderer.java}{be.hogent.captchabuilder.elementcreator.renderer.gimpy.BlockGimpyRenderer}
\codefragment{source/be/hogent/captchabuilder/elementcreator/renderer/gimpy/DropShadowGimpyRenderer.java}{be.hogent.captchabuilder.elementcreator.renderer.gimpy.DropShadowGimpyRenderer}
\codefragment{source/be/hogent/captchabuilder/elementcreator/renderer/gimpy/FishEyeGimpyRenderer.java}{be.hogent.captchabuilder.elementcreator.renderer.gimpy.FishEyeGimpyRenderer}
\codefragment{source/be/hogent/captchabuilder/elementcreator/renderer/gimpy/GimpyRenderer.java}{be.hogent.captchabuilder.elementcreator.renderer.gimpy.GimpyRenderer}
\codefragment{source/be/hogent/captchabuilder/elementcreator/renderer/gimpy/GimpyRendererBuilder.java}{be.hogent.captchabuilder.elementcreator.renderer.gimpy.GimpyRendererBuilder}
\codefragment{source/be/hogent/captchabuilder/elementcreator/renderer/gimpy/RippleGimpyRenderer.java}{be.hogent.captchabuilder.elementcreator.renderer.gimpy.RippleGimpyRenderer}
\codefragment{source/be/hogent/captchabuilder/elementcreator/renderer/gimpy/ShearGimpyRenderer.java}{be.hogent.captchabuilder.elementcreator.renderer.gimpy.ShearGimpyRenderer}
\codefragment{source/be/hogent/captchabuilder/elementcreator/renderer/gimpy/StretchGimpyRenderer.java}{be.hogent.captchabuilder.elementcreator.renderer.gimpy.StretchGimpyRenderer}
\codefragment{source/be/hogent/captchabuilder/elementcreator/renderer/text/AbstractWordRenderer.java}{be.hogent.captchabuilder.elementcreator.renderer.text.AbstractWordRenderer}
\codefragment{source/be/hogent/captchabuilder/elementcreator/renderer/text/ColoredEdgesWordRenderer.java}{be.hogent.captchabuilder.elementcreator.renderer.text.ColoredEdgesWordRenderer}
\codefragment{source/be/hogent/captchabuilder/elementcreator/renderer/text/DefaultWordRenderer.java}{be.hogent.captchabuilder.elementcreator.renderer.text.DefaultWordRenderer}
\codefragment{source/be/hogent/captchabuilder/elementcreator/renderer/text/WordRenderer.java}{be.hogent.captchabuilder.elementcreator.renderer.text.WordRenderer}
\codefragment{source/be/hogent/captchabuilder/elementcreator/renderer/text/WordRendererBuilder.java}{be.hogent.captchabuilder.elementcreator.renderer.text.WordRendererBuilder}
\codefragment{source/be/hogent/captchabuilder/util/enums/producer/BackgroundProducerType.java}{be.hogent.captchabuilder.util.enums.producer.BackgroundProducerType}
\codefragment{source/be/hogent/captchabuilder/util/enums/producer/BorderProducerType.java}{be.hogent.captchabuilder.util.enums.producer.BorderProducerType}
\codefragment{source/be/hogent/captchabuilder/util/enums/producer/NoiseProducerType.java}{be.hogent.captchabuilder.util.enums.producer.NoiseProducerType}
\codefragment{source/be/hogent/captchabuilder/util/enums/producer/TextProducerType.java}{be.hogent.captchabuilder.util.enums.producer.TextProducerType}
\codefragment{source/be/hogent/captchabuilder/util/enums/renderer/GimpyRendererType.java}{be.hogent.captchabuilder.util.enums.renderer.GimpyRendererType}
\codefragment{source/be/hogent/captchabuilder/util/enums/renderer/WordRendererType.java}{be.hogent.captchabuilder.util.enums.renderer.WordRendererType}
\codefragment{source/be/hogent/captchasolvingnetwork/network/encog/util/PropagationType.java}{be.hogent.captchasolvingnetwork.network.encog.util.PropagationType}


\bibliographystyle{plainnat}
\bibliography{Solving CAPTCHA using neural networks}

%%---------- Back matter -------------------------------------------------

\listoffigures
\listoftables

% Lijst van broncode-listings. Kan je weglaten als je dit niet nodig hebt.
\lstlistoflistings

\end{document}